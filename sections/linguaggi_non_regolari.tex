\section{Linguaggi Non Regolari}
Di quanti stati necessita l'automa che riconoscere linguaggio $\{0^n1^n|n\geq0\}$?
Occorrono $2n$ stati per poter riconoscere il linguaggio. Essendo $n$ infinito non è possibile 
determinare un numero finito di stati, pertanto il linguaggio $\{0^n1^n|n\geq0\}$ non è regolare, 
proprio perché non può essere riconosciuto da un automa a stati finiti. 

\paragraph{Dimostrazione}

È necessario ragionare \textit{per assurdo}, in quanto siamo costretti a dimostrare che non esiste un automa 
a stati finiti che riconosce un linguaggio che ipotizziamo essere non regolare. 
Supponiamo che  $L_{01}=\{0^n1^n|n\geq0\}$ sia regolare, allora esiste un DFA $A$ che accetta $L_{01}$ con $k$ stati. 
Cerchiamo ora di dimostrare che esiste una parola appartenente a $L_{01}$ che non viene riconosciuta dall'automa DFA.
L'automa segue una computazione $r_0, r_1, r_2,\dots,r_k$ ovvero di lunghezza $k+1$, questo comporta che 
all'interno della sequenza c'è uno stato che si ripete, supponiamo che si tratti di  $r_i =r_j$ per qualche $a$.
Posso sfruttare questo fatto per far sbagliare l'automa:
considero la parola $0^i1^i$, la computazione ha la forma $$r_0, r_1, \dots, r_i, s_1, \dots, s_i$$
$s_i$ è uno stato finale? Sì, $s_i$ deve essere finale. 
Cosa succede se a questo automa forniamo in input la parola $0^j1^i$ ? 
Abbiamo detto che $r_j=r_i$ quindi l'automa terminerà nello stesso stato finale $s_i$ e 
dunque verrebbe riconosciuta una parola che non appartiene a $L_{01}$.

\subsection{SBRODEGHI}
\begin{itemize}
	\item supponiamo che $l_{0}=\{0^n1^n | n \geq 0\}$ 
	\item ...

	\item cosa succede quando l; automa $A$ legge $1^i$ partendo da $q$ 
	\item se l'automa finisce la lettura in uno stato finale
		\item allora accetta, sbagliando la parola $0^j1^j$ 
	\item se l'automa finisce la lettura in uno stato non finale
		\item allora rifiuta sbagliando la parola $0^i1^i$ 
	\item in entrambi i casi abbiamo ingannato l'automa, quindi $L_{01}$ *non può essere regolare*
\end{itemize}

\subsubsection{Proprietà dei linguaggi regolari}
Possono essere utilizzate per dimostrare che un DFA effettua un loop per accettare un linguaggio regolare.
La dimostrazione è più facile rispetto a quella precedente. 
\subsection{Pumping Lemma}
Sia $L$ un linguaggio regolare. Allora
\begin{itemize}
	\item  \g{esiste una lunghezza} $k>0$ tale che 
	\item  \g{ogni parola} $w\in L$ di lunghezza $|w|\geq k$ 
	\item  \g{può essere spezzata} in $w =xyz$ tale che 
\end{itemize}
\begin{enumerate}
	\item $y\neq \varepsilon$ (il secondo pezzo è non vuoto)
	\item $|xy|\leq k$ (i primi due pezzi sono lunghi al max $k$)
	\item $\forall i \geq 0, xy^iz\in L$ (possiamo "pompare" $y$ rimandandolo in $L$)
\end{enumerate}
\subsubsection{Dimostrazione}
\begin{itemize}
	\item Supponiamo che $L$ sia un linguaggio regolare
	\item Allora è riconosciuto da un DFA con, supponiamo, $k$ stati
	\item Consideriamo una parola $w = a_1, a_2, \dots, a_n\in L$ di lunghezza $n\geq k$ 
	\item Consideriamo gli stati nella computazione di $A$ per $w$ 
\end{itemize}
$$p_0, p_1, p_2,\dots p_k\dots p_n$$
Siccome in $p_0, p_1,\dots p_k$ ci sono $k+1$ stati ne esiste uno che si ripete:\\
Esistono $l< m$ tali che $p_l=p_m$ e $m \leq k$ 

\subsubsection{Pumping lemma come gioco} 
Esiste una strategia che ci consente di vincere sempre su un certo linguaggio. 
\begin{enumerate}
	\item $\exists k>o$  (giocatore 1 sceglie la dimensione di $k$)
	\item $\forall w\in L, |w|\geq k$ (giocatore 2 sceglie una parola)
	\item $\exists xyz\mid w= xyz$
		$y\neq \varepsilon$ 
		$|xy| \leq k$ (giocatore 1 sceglie come partizionare la parola)
	\item	$\forall i\geq 0 xy^iz\in L$ (giocatore 2 sceglie una potenza affinché la condizione sia verificata)
\end{enumerate}
Se giocatore 2 vince allora il linguaggio non è regolare. 

\subsubsection{Esistono linguaggi non regolari che rispettano il Pumping Lemma}

\paragraph{Sia $L_{ab}$ il linguaggio delle stringhe sull'alfabeto $(a,b)$ dove il numero di $a$ è uguale al numero di $b$. $L_{ab}$ è regolare?}\nin
Per assurdo ipotizzo $L_{ab}$ regolare. \\
Se lo è, allora rispetta il Pumping Lemma $\rightarrow$ deve esistere una lunghezza $k>0$ che rende vero il Pumping Lemma.\\
Consideriamo la parola $w=a^kb^k$ , $w$ appartiene al linguaggio e $|w|>k$.\\
Per ogni suddivisione $w\in xyz$ dove $y\neq \varnothing$ e $|xy|\leq k$ \\
$$x = a^P$$
$$y = a^Q$$
$$z = a^{K-P-Q}b^K$$
Se prendiamo come esponente $i=2$ , $xy^2z=a^Pa^{2Q}a^{K-P-Q}b^K= a^{Q+K}b^K$ la parola non fa parte del linguaggio.\\
\g{Conclusione: per assurdo non è regolare.}\\

\paragraph{Il linguaggio $L_{rev} = \{ww^R:w \in\{a,b\}*\}$ è regolare?}\nin\\

$w^R = w$ scritto alla rovescia

$w = w_1, w_2, \dots, w_k$ 

$w^R=w_n, w_{n-1}, \dots, w_1$ 
\begin{enumerate}
	\item Suppongo $L_{rev}$ sia regolare,allora esiste $k>0$ che rende vero il Pumping Lemma
	\item considero la parola $w=a^kb^2a^k$ 
	\item Per ogni suddivisione di $xyz$ dove $y\neq \varepsilon$ e $|xy|\leq k$ 
\end{enumerate}
$$x = a^P$$
$$y = a^Q$$
$$z = a^{K-P-Q}b^2a^K$$
Se prendiamo come esponente $i=2$ , $xy^2z=a^Pa^{2Q}a^{K-P-Q}b^2a^K= a^{Q+K}b^2a^K$ la parola non fa parte del linguaggio.\\
\g{Conclusione: per assurdo non è regolare.}


\paragraph{Il linguaggio $L_{p} = \{1^p: p$  è primo$\}$ è regolare?}\nin
\begin{enumerate}
	\item Assumo che $L_p$ sia regolare e che $k>0$ sia la lunghezza che rende vero il Pumping Lemma. 
	\item Considero $w=1j$ dove $j$ è il primo numero primo $>k$ 
	\item Per ogni suddivisione $w=xyz$ dove $y\neq \varnothing$ e $|xy| \leq k$ 
		\begin{enumerate}
			\item $x=1^P$ 
			\item $y = 1^Q$ 
			\item $z=1^{J-P-Q}$ dove $Q>0$ e $P+Q \leq k$ 
		\end{enumerate}
\end{enumerate}

$i=2$ \\
$xy^2z = 1^P1^{2Q}1^{J-P-Q}=1^{Q+J}$\\
\\
$i=2J$\\
$xy^{2J}z = 1^P1^{2jQ}1^{J-P-Q}=1^{(2J)Q+J-Q}$\\
$= 1^{(2J-1)Q+J} = 1^{(Q+1)J} \not\in L_p$ \\
La parola non è contenuta nel linguaggio in quanto la sua lunghezza è un numero scomponibile in fattori. 


