\section{Classe P}
Riassunto: 
\begin{itemize}
  \item Differenza di tempo \g{polinomiale} tra TM a nastro singolo e multi-nastro
  \item Differenza di tempo \g{esponenziale} tra TM deterministichee non deterministiche
\end{itemize}
Una differenza \g{polinomiale} è considerata piccola
\begin{itemize}
  \item Tutti i modelli di calcolo deterministici ``ragionevoli'' sono 
    \g{polinomialmente equivalenti}
  \item ``Ragionevole'' è definito in modo approssimativo, ma include
    modelli che assomigliano molto ai computer reali
\end{itemize}
Una differenza \g{esponenziale} è considerata grande
\begin{definition}
  $P$ è l classe di linguaggi che sono decidibili in \g{tempo polinomiale}
  da una TM deterministica a singolo nastro
  \begin{displaymath}
    P=\underset{k}{\bigcup}TIME(n^k)
  \end{displaymath}
\end{definition}
\begin{itemize}
  \item $P$ è invariante per i modelli di calcolo \g{polinomialmente equivalenti}
    ad una TM deterministica
  \item $P$ corrisponde approssimativamente ai problemi che sono 
    \g{realisticamente risolvibili} da un computer
\end{itemize}
Per dimostrare che un problema/algoritmo è in $P$ 
\begin{itemize}
  \item Descrivi l'algoritmo per fasi numerate 
  \item Dai un limite superiore polinomiale al numero di fasi che l'algoritmo 
    esegue per un input di lunghezza $n$ 
  \item Assicurati che ogni fase possa essere completata in tempo polinomiale
    su un modello di calcolo deterministico ragionevole
  \item L'input deve essere codificato in modo ragionevole
\end{itemize}
\subsection{Due problemi in $P$}
\begin{description}
  \item[Raggiungibilità di un grafo] 
    $PATH = \{\langle G,s,t\rangle\mid G\textrm{ grafo che contiene un cammino da} s \textrm{ a }t\}$
  \item[Numeri relativamente primi]
    $RELPRIME = \{(x,y)\mid 1\textrm{ è il massimo comun divisore di }x \textrm{ e }y\}$
\end{description}
\begin{theorem}
  ogni linguaggi context-free è un elemento in $P$ 
\end{theorem}
\begin{itemize}
  \item Abbiamo dimostrato che ogni CFL è \g{decidibile}
    \begin{itemize}
      \item L'algoritmo nella dimostrazione è \g{esponenziale}
    \end{itemize}
  \item La soluzione polinomiale usa la \g{programmazione dinamica}
  \item La complessità è $O(n^3)$ 
\end{itemize}

